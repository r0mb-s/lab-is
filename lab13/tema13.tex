\documentclass{article}

\usepackage{enumerate}
\usepackage{amsmath}
\usepackage{amsfonts}

\begin{document}

\section{Explain rigorously why each of these triples hold:}

\begin{enumerate}[(a)]
	\item \{\{$x == y$\}\} $z := x - y$ \{\{$z == 0$\}\}
	      \\ We assume that x equals y. If x equals y, then x subtracted by y will equal 0. As z equals
	      x subtracted by y, z equals 0 - which is the postcondition
	\item \{\{true\}\} $x := 100$ \{\{$x == 100$\}\}
	      \\ As the term true is always true we can ignore the precondition. If x equals 100, then
	      x will equal 100 - which is the postcondition
	\item \{\{$0 \le x < 100$\}\} x := x + 1 \{\{$0 \le x \le 100$\}\}
	      \\ As x $\in [0, 100)$, x + 1 $\in [0 + 1, 100 + 1)$ which is equivalent with
	      x + 1 $\in [1, 101)$ which is equivalent with x + 1 $\in [0, 100]$ - which is the
	      postcondition
\end{enumerate}

\section{For each of the following triples, find initial values for x and y that demonstrate that the
  triple does not hold.}
\begin{enumerate}[(a)]
	\item \{\{true\}\} $x := 2 * y$ \{\{$y \le x$\}\}
	      \\ When y equals -1, x equals -2. As -2 is less than -1, $y > x$ - which contradicts the
	      postcondition
	\item \{\{$0 \le x$\}\} $x := x - 1$ \{\{$0 \le x$\}\}
	      \\ When $x = 0$, $x - 1 = -1$ and as $-1 < 0$ the postcondition doesn't stand
\end{enumerate}

\section{For each of the following triples, come up with some predicate to replace the question mark
  to make it a Hoare triple that holds. Make your conditions as precise as possible.}
\begin{enumerate}[(a)]
	\item \{\{$0 \le x < 100$\}\} $x := 2 * x$ \{\{?\}\}
	      \\ As the program states that x will become $2 * x$, then the inequalities from the precondition
	      will become $0 \le x < 200$ which can become the postcondition
	\item \{\{$0 \le x < N$\}\} $x := x + 1$ \{\{?\}\}
	      \\ As the program states that x will become $x + 1$, then the inequalities from the precondition
	      will become $1 \le x < N + 1$ which can become the postcondition
\end{enumerate}

\section{For each of the following triples, come up with some predicate to replace the question mark
  to make it a Hoare triple that holds. Make your conditions as precise as possible.}
\begin{enumerate}[(a)]
	\item \{\{?\}\} $x := 400$ \{\{$x == 400$\}\}
	      \\ As we need x to be 400, and the program states that x will be 400, the precondition can
	      be anything
	\item \{\{?\}\} $x := 65$ \{\{$y \le x$\}\}
	      \\ If $x = 65$ then $y \le 65$, which should be the precondition
\end{enumerate}

\section{Write the program which computes the sum of first n natural numbers.}

For the following Hoare Logic formula we have:
\\ P: $n \in \mathbb{N}$
\\ I: $\text{sum} == \frac{i(i + 1)}{2}$
\\ b: $i \le n$
\\ c: $\text{sum} = \text{sum} + 1$, $i = i + 1$
\\ Q: $\text{sum} = \frac{n(n + 1)}{2}$
\\ t: $n - 1$

\begin{equation*}
    \frac
    {P \Longrightarrow I \quad I \Longrightarrow t \ge 0 \quad \{\{I \wedge b \wedge t\}\} \ c \ \{\{I \wedge t < N\}\} \quad (I \wedge \neg b) \Longrightarrow Q}
    {\{\{P\}\} \ \text{while} \ b \ \text{do} \ c \ \{\{Q\}\}}
\end{equation*}

\begin{verbatim}
method sum(n: nat) returns (sum: nat)
  ensures sum == n * (n + 1) / 2
{
  var i := 0;
  sum := 0;
  while i <= n
    invariant 0 <= i <= n + 1
    invariant sum == i * (i - 1) / 2
    decreases n - i
  {
    sum := sum + i;
    i := i + 1;
  }
}
\end{verbatim}

\section{Write the program which computes the product of first n natural numbers. Prove its total
correctness.}
\begin{verbatim}
function factorial(n: nat) : nat
  decreases n
{
  if n == 0 then 1 else n * factorial(n - 1)
}

method product(n: nat) returns (prod: nat)
  ensures prod == factorial(n)
{
  var i := 1;
  prod := 1;
  while i <= n
    invariant 0 <= i <= n + 1
    invariant prod == factorial(i - 1)
    decreases n - i
  {
    prod := prod * i;
    i := i + 1;
  }
}
\end{verbatim}

\end{document}
